\documentclass[12pt]{article}
\usepackage[margin=1in]{geometry} 
\usepackage{amsmath,amsthm,amssymb,scrextend}
\usepackage{fancyhdr}
\setlength{\headheight}{14.5pt}
\addtolength{\topmargin}{-2.5pt}
\pagestyle{fancy}

\newcommand{\cont}{\subseteq}
\usepackage{tikz}
\usepackage{pgfplots}
\usepackage{amsmath}
\usepackage[mathscr]{euscript}
\let\euscr\mathscr\let\mathscr\relax% just so we can load this and rsfs
\usepackage[scr]{rsfso}
\usepackage{amsthm}
\usepackage{amssymb}
\usepackage{multicol}
\usepackage{bbm}
\usepackage[colorlinks=true, pdfstartview=FitV, linkcolor=blue,
citecolor=blue, urlcolor=blue]{hyperref}

\DeclareMathOperator{\arcsec}{arcsec}
\DeclareMathOperator{\arccot}{arccot}
\DeclareMathOperator{\arccsc}{arccsc}
\newcommand{\ddx}{\frac{d}{dx}}
\newcommand{\dfdx}{\frac{df}{dx}}
\newcommand{\ddxp}[1]{\frac{d}{dx}\left(#1 \right)}
\newcommand{\dydx}{\frac{dy}{dx}}
\newcommand{\pfpt}{\dfrac{\partial f_t}{\partial t}}
\newcommand{\pfps}{\dfrac{\partial f_t}{\partial S_t}}
\newcommand{\pffpss}{\dfrac{\partial^2 f_t}{\partial S_t^2}}
\newcommand{\ind}{\mathbbm{1}}
\let\ds\displaystyle
\newcommand{\intx}[1]{\int#1 \, dx}
\newcommand{\intt}[1]{\int#1 \, dt}
\newcommand{\defint}[3]{\int_{#1}^{#2} #3 \, dx}
\newcommand{\imp}{\Rightarrow}
\newcommand{\un}{\cup}
\newcommand{\inter}{\cap}
\newcommand{\ps}{\mathscr{P}}
\newcommand{\set}[1]{\left\{ #1 \right\}}
\newtheorem*{sol}{Solution}
\newtheorem*{claim}{Claim}
\newtheorem{problem}{Problem}
\pgfplotsset{compat=1.17}
\begin{document}
 
% Don't change the above session

\lhead{Yu-Chen Den}
\chead{Financial Engineering Midterm}
\rhead{\today}
 
% \maketitle
\section{Binomial Model}
\subsection*{one-period standard Eurpoean Call option Binomial model}
Risk Neutral probability $q$ is derived as
\begin{align*}
    q = \dfrac{e^{rT}- d}{u-d}
\end{align*}
So, the price of the option is
\begin{align*}
    f = e^{-rT}\Big[q\times \max(uS_0-K, 0) + (1-q)\times \max(dS_0 -K, 0)\Big]
\end{align*}
\subsection*{Generalized European Call option Binomial Model}
$n$ periods, where $\Delta t = \dfrac{T}{n}$ 
\begin{align*}
    f_u &= e^{-r\Delta t}\Big[q\times f_{uu} + (1-q)\times f_{ud}\Big],\ f_d = e^{-r\Delta t}\Big[q\times f_{du} + (1-q)\times f_{dd}\Big]\\
    f &= e^{-r\Delta t}\Big[q\times f_u + (1-q)\times f_d\Big] = e^{-2r\Delta t}\Big[q^2\times f_{uu} + 2q(1-q)\times f_{ud} + (1-q)^2\times f_{dd}\Big]
\end{align*}
So, when $n = 3$, we have
\begin{align*}
    f = e^{-3r\Delta t}\Big[q^3 f_{C_3^3 u^3 d^0} + 3q^2(1-q)f_{C^3_2 u^2 d^1} + 3q(1-q)^2 f_{C^3_1 u^1 d^2} + (1-q)^3 f_{C^3_0 u^0d^3}\Big]
\end{align*}
As we expand to $n$ periods, we will have
\begin{align*}
   f &= e^{-nr\Delta t}\Big[C^n_n q^n(1-q)^0 f_{C_n^n u^n d^0} + C^n_{n-1}q^{n-1}(1-q)^1f_{C_{n-1}^{n} u^{n-1} d^1} + \cdots + C^n_0 q^0(1-q)^n f_{C_0^n u^0d^n}\Big]\\
   & = e^{-nr\Delta t}\Big[\sum_{i=0}^n C^n_i q^i(1-q)^{n-i} \max(u^id^{n-i}S_0-K, 0)\Big]\\
   & = e^{-nr\Delta t}\Big[\sum_{i=0}^n C^n_i q^i(1-q)^{n-i} \max(u^id^{n-i}S_0-K, 0)\Big]
\end{align*} 
Assume that $a$ is the minimum number of upward moves that the stock must finish in-the-money.\\\\
$\forall i < a,\ \max(u^id^{n-i}S_0-K, 0) = 0$\\\\
$\forall i \geq a,\ \max(u^id^{n-i}S_0-K, 0) = u^id^{n-i}S_0-K$\\\\
So, 
\begin{align*}
   f & = e^{-nr\Delta t}\Big[\sum_{i=a}^n C^n_i q^i(1-q)^{n-i} (u^id^{n-i}S_0-K)\Big]\\
   & = S\Big[\sum_{i=a}^n C^n_i (que^{-r\Delta t})^i[(1-q)de^{-r\Delta t}]^{n-i}\Big]- Ke^{-nr\Delta t}\sum_{i=a}^n C^n_i q^i(1-q)^{n-i}
\end{align*}
Let $q' = que^{-r\Delta t}$, $(1-q)' = (1-q)de^{-r\Delta t}$
\begin{align*}
    f &= S\Big[\sum_{i=a}^n C^n_i q'^i[(1-q)']^{n-i}\Big]- Ke^{-nr\Delta t}\sum_{i=a}^n C^n_i q^i(1-q)^{n-i}\\
    & = SP(X\ge a) - Ke^{-rT}P(Y\ge a)
\end{align*}
Where $X\sim Bin(n, q')$, $Y\sim Bin(n, q)$
\subsection*{CRR to BS}
According to Geometric Brownian Motion (GBM), we have\\
\begin{align*}
    \dfrac{S_t}{S} = r dt + \sigma dW_t \leftrightarrow \ln(\dfrac{S_T}{S}) = (r-\dfrac{\sigma^2}{2})T + \sigma W_T
\end{align*}
Then the mean and variance of stock return can be present as
\begin{align*}
    E\left[\ln(\dfrac{S_T}{S})\right] = \left(r-\dfrac{1}{2}\sigma^2\right)T\\\\
    Var\left[\ln(\dfrac{S_T}{S})\right] = \sigma^2 T
\end{align*}
Based on CRR, $S_T = Su^I d^{N-I}\Rightarrow \ln(\dfrac{S_T}{S}) = \ln(\dfrac{Su^I d^{N-I}}{S}) = N\ln d + I\ln(\dfrac{u}{d})$, Let $I = Nq$
\begin{align*}
    E\left[\ln\left(\dfrac{S_T}{S}\right)\right] = N\ln d + Nq\ln(\dfrac{u}{d}) \equiv \hat{\mu}N\\\\
    Var\left[\ln\left(\dfrac{S_T}{S}\right)\right] = Nq(1-q)\ln^2(\dfrac{u}{d}) \equiv \hat{\sigma}^2N
\end{align*}
To converge the pricing formula, it must satisfy the following properties
\begin{enumerate}
    \item[(i)] $\hat{\mu}N \to \left(r-\dfrac{1}{2}\sigma^2\right)T$ as $n\to\infty$\\
    \item[(ii)] $\hat{\sigma}^2N \to \sigma^2 T$ as $n\to\infty$\\
\end{enumerate}
Furthermore, we set $u = e^{\sigma\sqrt{\Delta t}},\ \ d = e^{-\sigma\sqrt{\Delta t}}$. According to above, we devide the model into 2 parts, $P(X\ge a)$ and $P(Y \ge a)$\\
\begin{align*}
    1-P(Y\ge a) = P(Y\leq a) = P(\dfrac{Y-Nq}{\sqrt{Nq(1-q)}}\leq \dfrac{(a-1)-Nq}{\sqrt{Nq(1-q)}})\\
\end{align*}
Because $u^a d^{N-a}S-K> 0$, so $a> \ln(\dfrac{K}{Sd^N})/\ln(\dfrac{u}{d})$. And there exists a $\varepsilon\in [0,1]$, such that $a-1 = \ln(\dfrac{K}{Sd^N})/\ln(\dfrac{u}{d})-\varepsilon$. Furthermore, from above, we know that
\begin{align*}
    \ln d = \hat{\mu} - q\ln(\dfrac{u}{d})\\
    \ln(\dfrac{u}{d}) = \dfrac{\hat{\sigma}}{\sqrt{q(1-q)}}\\
\end{align*}
So we can replace $a-1$ by them and denote as
\begin{align*}
    \dfrac{(a-1)-Nq}{\sqrt{Nq(1-q)}} & = \dfrac{(\ln(\dfrac{K}{Sd^N})/\ln(\dfrac{u}{d})-\varepsilon)-Nq}{\sqrt{Nq(1-q)}}\\
    & = \dfrac{\ln(\dfrac{K}{Sd^N})/\ln(\dfrac{u}{d})}{\sqrt{Nq(1-q)}} - \dfrac{\varepsilon+Nq}{\sqrt{Nq(1-q)}}\\
\end{align*}
From above, we can know that $\dfrac{1}{\hat{\sigma}} = \dfrac{\sqrt{q(1-q)}}{\ln(\dfrac{u}{d})}$
\begin{align*}
    \dfrac{(a-1)-Nq}{\sqrt{Nq(1-q)}} & = \dfrac{\ln(\dfrac{K}{S})-N\ln d}{\sqrt{N}\hat{\sigma}} - \dfrac{\varepsilon+Nq}{\sqrt{Nq(1-q)}} = \dfrac{\ln(\dfrac{K}{S})-N(\hat{\mu} - q\ln(\dfrac{u}{d}))}{\sqrt{N}\hat{\sigma}} - \dfrac{\varepsilon+Nq}{\sqrt{Nq(1-q)}}\\
    & = \dfrac{\ln(\dfrac{K}{S})-N(\hat{\mu} - q\ln(\dfrac{u}{d}))}{\sqrt{N}\hat{\sigma}} - \dfrac{\ln(\dfrac{u}{d})(\varepsilon+Nq)}{\sqrt{N}\hat\sigma}\\
    & = \dfrac{\ln(\dfrac{K}{S})-N\hat{\mu}-\ln(\dfrac{u}{d}\varepsilon)}{\sqrt{N}\hat{\sigma}}\\
\end{align*}
Follow the below properties
\begin{enumerate}
    \item[(i)] $\hat{\mu}N \to (r-\dfrac{1}{2}\sigma^2)T$ as $N\to\infty$
    \item[(ii)] $\hat{\sigma^2}N\to\sigma^2 T$ as $N\to\infty$
    \item[(iii)] $\ln(\dfrac{u}{d}) = \ln(\dfrac{e^{\sigma\sqrt{\Delta t}}}{e^{-\sigma\sqrt{\Delta t}}}) = 2\sigma\sqrt{\Delta t}\to 0$ as $N\to\infty$\\   
\end{enumerate}
Adopt the Central Limit Theorem (CLT)
\begin{align*}
    1-P(Y\ge a) = P(\dfrac{Y-Nq}{\sqrt{Nq(1-q)}}\leq \dfrac{(a-1)-Nq}{\sqrt{Nq(1-q)}}) \to P(Z \leq \dfrac{\ln(K/S) - (r-\dfrac{1}{2}\sigma^2)T}{\sigma\sqrt{T}})
\end{align*}
So
\begin{align*}
    P(Y\ge a) \to P(Z \leq \dfrac{\ln(S/K) + (r-\dfrac{1}{2}\sigma^2)T}{\sigma\sqrt{T}}) = N(d_2)
\end{align*}
We use the same method on $P(X\ge a) \to N(d_1)$
\section{Stochastic Process and It\^{o}'s Lemma}
\subsection*{It\^{o} Process}
\begin{align*}
    df_t &= \Big(\pfpt + \dfrac{\partial f_t}{\partial X_{1,t}}\mu_1 X_{1,t} + \dfrac{\partial f_t}{\partial X_{2,t}}\mu_2 X_{2,t} + \dfrac{1}{2} \dfrac{\partial^2 f_t}{\partial X^2_{1,t}}\sigma^2_1 X^2_{1,t} + \dfrac{1}{2} \dfrac{\partial^2 f_t}{\partial X^2_{2,t}}\sigma^2_2 X^2_{2,t}\\
    &\ \ \ \ + \dfrac{\partial^2 f_t}{\partial X_{1,t}\partial X_{2,t}} \sigma_1\sigma_2 X_{1,t}X_{2,t}\rho_{12}\Big)dt + \dfrac{\partial f_t}{\partial X_{1,t}}\sigma_1 X_{1,t}dW_{1,t} + \dfrac{\partial f_t}{\partial X_{2,t}}\sigma_2 X_{2,t}dW_{2,t}
\end{align*}
\subsection*{Derive It\^{o}'s Lemma}
$f_t = \dfrac{X_{1,t}}{X_{2,t}}$, $\ \ $$dX_{1,t} = \mu_1 X_{1,t} + \sigma X_{1,t}dW_{1,t}$,$\ \ $$dX_{2,t} = \mu_2 X_{2,t} + \sigma X_{2,t}dW_{2,t}$\\\\
We can first write down\\\\
$\pfpt = 0,\ \ \dfrac{\partial f_t}{\partial X_{1,t}} = \dfrac{1}{X_{2,t}},\ \ \dfrac{\partial f_t}{\partial X_{2,t}} = -\dfrac{X_{1,t}}{X^2_{2,t}},\ \ \dfrac{\partial^2 f_t}{\partial X^2_{1,t}} = 0,\ \ \dfrac{\partial^2 f_t}{\partial X^2_{2,t}} = \dfrac{2X_{1,t}}{X^3_{2,t}},\ \ \dfrac{\partial^2 f_t}{\partial X_{1,t}\partial X_{2,t}} = -\dfrac{1}{X^2_{2,t}}$\\\\
So
\begin{align*}
    df_t &= \Big(\dfrac{1}{X_{2,t}}\mu_1 X_{1,t} -\dfrac{X_{1,t}}{X^2_{2,t}}\mu_2 X_{2,t} + \dfrac{1}{2}\dfrac{2X_{1,t}}{X^3_{2,t}}\sigma_2^2X^2_{2,t} -\dfrac{1}{X^2_{2,t}}\sigma_1\sigma_2 X_{1,t}X_{2,t}\rho_{12}\Big)dt\\
    &\ \ \ \ + \dfrac{1}{X_{2,t}}\sigma_1 X_{1,t}dW_{1,t} -\dfrac{X_{1,t}}{X^2_{2,t}}\sigma_2 X_{2,t}dW_{2,t}\\
    & = \dfrac{X_{1,t}}{X_{2,t}}\Big[\Big(\mu_1 + \mu_2 + \sigma^2_2 - \sigma_1 \sigma_2 \rho_{12}\Big)dt + \sigma_1 dW_{1,t} - \sigma_2 dW_{2,t}\Big]
\end{align*}
Then
\begin{align*}
    \dfrac{df_t}{f_t} = \Big(\mu_1 + \mu_2 + \sigma^2_2 - \sigma_1 \sigma_2 \rho_{12}\Big)dt + \sigma_1 dW_{1,t} - \sigma_2 dW_{2,t}
\end{align*}
\newpage
\section{Black-Scholes Model and Merton Pricing Formula}
\subsection*{Assumptions of Black-Scholes Model}
\begin{enumerate}
    \item Stock Price follows It\^{o} process $dS_t = \mu S_tdt + \sigma S_tdW_t$, under this assumption, we can derive that the stock price follows logarithmic normal distribution, and both $\mu$, $\sigma$ are constant.
    \item Market is frictioness. There are no transcation costs, no tax, no liquidity limitation, and no law restrictions.
    \item Stock trading in the market is continuous.
    \item We can short stocks without any restriction.
    \item No underlying dividend.
    \item Options are European.
    \item risk free rate $r$ exists. 
\end{enumerate}
\subsection*{Black-Scholes PDE}
Assume the stock change process is $dS_t = \mu S_tdt + \sigma S_tdW_t$\\\\
Let $f_t$ be the price of a European call option\\\\
Then
\begin{align*}
    df_t = (\pfpt + \pfps \mu S_t + \dfrac{1}{2}\pffpss \sigma^2 S_t^2)dt + \pfps \sigma S_tdW_t
\end{align*}
Now, construct a risk-free investment portfolio $\pi_t$, where $\pi_t = -f_t + \pfps S_t$, So $d\pi_t = -df_t + \pfps dS_t$
\begin{align*}
    d\pi_t &= -(\pfpt + \pfps \mu S_t + \dfrac{1}{2}\pffpss \sigma^2 S_t^2)dt - \pfps \sigma S_tdW_t + \pfps dS_t\\
    &= -(\pfpt + \pfps \mu S_t + \dfrac{1}{2}\pffpss \sigma^2 S_t^2)dt - \pfps \sigma S_tdW_t + \pfps (\mu S_tdt + \sigma S_tdW_t)\\
    &= -(\pfpt + \dfrac{1}{2}\pffpss \sigma^2 S_t^2)dt = r\pi_tdt = r(-f_t + \pfps S_t)dt
\end{align*}
Both devided by $dt$
\begin{align*}
    -(\pfpt + \dfrac{1}{2}\pffpss \sigma^2 S_t^2) = r(-f_t + \pfps S_t)
\end{align*}
\begin{align*}
    rf_t = \pfpt + \dfrac{1}{2}\pffpss \sigma^2 S_t^2 + r\pfps S_t
\end{align*}
\subsection*{Feynman-Kac Thoerem}
According to the above, we know that $rf_t = \pfpt + \dfrac{1}{2}\pffpss \sigma^2 S_t^2 + r\pfps S_t$\\\\
Let $f_T = \max(S_T - K, 0) = (S_T-K)\ind_{S_T>K}\leftrightarrow f_t = E_Q[e^{-r(T-t)}f_T|\mathcal{F}_t]$\\
\begin{align*}
    f_t &= e^{-r(T-t)}E_Q[f_T|\mathcal{F}_t] = e^{-r(T-t)}E_Q[(S_T-K)\ind_{S_T>K}|\mathcal{F}_t]\\\\
    &= e^{-r(T-t)}E_Q[S_T\ind_{S_T>K}] - Ke^{-r(T-t)}E_Q[\ind_{S_T>K}|\mathcal{F}_t]
\end{align*}\\
Assume we set $E_1 = e^{-r(T-t)}E_Q[S_T\ind_{S_T>K}]$ and $E_2 = Ke^{-r(T-t)}E_Q[\ind_{S_T>K}|\mathcal{F}_t]$\\\\
And we know $S_T = S_t \exp[(r-\frac{1}{2}\sigma^2)(T-t)+\sigma\Delta W_{T-t}]$, $\Delta W_{T-t} = W_T - W_t \sim N(0, T-t)$\\\\
Thus, $S_T\sim \log N(\ln S_t + (r-\frac{1}{2}\sigma^2)(T-t), \sigma^2(T-t))$, Let $\ln S_t + (r-\frac{1}{2}\sigma^2)(T-t) = a$\\\\
, and $\sigma^2(T-t) = b^2$\\\\
Last, let $S_T = Y = e^x$. That is, $Y = e^x \sim \log N(a, b^2)$, $x = \log Y \sim N(a, b^2)$
\begin{align*}
    E_Q[e^x\ind_{x>\ln K}|\mathcal{F}_t] &= \int_{\ln K}^{\infty} e^xf(x)dx = \int_{\ln K}^{\infty} e^x \dfrac{1}{\sqrt{2\pi b^2}}\exp(-\dfrac{(x-a)^2}{2b^2})dx\\
    &= \int_{\ln K}^{\infty} e^x \dfrac{1}{\sqrt{2\pi b^2}}\exp(-\dfrac{1}{2b^2}(x^2 - 2ax + a^2))dx\\
    & = \int_{\ln K}^{\infty}\dfrac{1}{\sqrt{2\pi b^2}}\exp(-\dfrac{1}{2b^2}(x^2 -2(a+b^2)x + a^2))dx\\
    & = \int_{\ln K}^{\infty}\dfrac{1}{\sqrt{2\pi b^2}}\exp(-\dfrac{1}{2b^2}(x-(a+b^2)^2)dx\times \exp[\dfrac{1}{2b^2}(2ab^2 + b^4)]
\end{align*}
Because $\dfrac{1}{\sqrt{2\pi b^2}}\exp(-\dfrac{1}{2b^2}(x-(a+b^2)^2)$ means that $x\sim N(a+b^2, b^2)$\\
\begin{align*}
    E_Q[e^x\ind_{x>\ln K}|\mathcal{F}_t] & = p(x>\ln K)\times \exp[a+\dfrac{1}{2b^2}] = P(\dfrac{x-(a+b^2)}{b} > \dfrac{\ln K-(a+b^2)}{b})\times \exp[a+\dfrac{1}{2}b^2]\\
    & = P(Z > \dfrac{\ln K-(a+b^2)}{b})\times \exp[a+\dfrac{1}{2}b^2] = P(Z < \dfrac{(a+b^2)-\ln K}{b})\times \exp[a+\dfrac{1}{2}b^2]\\
    & = \Phi \left(\dfrac{(a+b^2)-\ln K}{b}\right)\times \exp[a+\dfrac{1}{2}b^2]
\end{align*}
We know that $a = \ln S_t + (r-\dfrac{1}{2}\sigma^2) (T-t)$, and $b^2 = \sigma^2 (T-t)$
\begin{align*}
    \Phi \left(\dfrac{(a+b^2)-\ln K}{b}\right) & = \Phi \left(\dfrac{\ln S_t + (r-\dfrac{1}{2}\sigma^2) (T-t) + \sigma^2 (T-t) - \ln K}{\sigma\sqrt{T-t}}\right)\\
    & = \Phi \left(\dfrac{\ln \dfrac{S_t}{K_t} +\dfrac{1}{2}\sigma^2(T-t)}{\sigma\sqrt{T-t}}\right) = N(d_1)
\end{align*}
\begin{align*}
    \exp [a+\dfrac{1}{2}b^2] = \exp [\ln S_t + (r-\dfrac{1}{2}\sigma^2)(T-t) + \dfrac{1}{2}\sigma^2(T-t)] = S_te^{r(T-t)}
\end{align*}
So,
\begin{align*}
    e^{-r(T-t)}E_Q[S_T\ind_{S_T>K}] = S_tN(d_1)
\end{align*}
$E_2$ has the same way to get it.\\
\begin{align*}
    E_Q[\ind{x>\ln K}|\mathcal{F}_t] = \int_{\ln K}^{\infty}\dfrac{1}{\sqrt{2\pi b^2}}\exp(-\dfrac{(x-a)^2}{2b^2})dx
\end{align*}
We know that $\dfrac{1}{\sqrt{2\pi b^2}}\exp(-\dfrac{(x-a)^2}{2b^2})$ means that $x\sim N(a, b^2)$\\
\begin{align*}
    E_Q[\ind{x>\ln K}|\mathcal{F}_t] &= P(x>\ln K) = P(\dfrac{x-a}{b} > \dfrac{\ln K - a}{b}) = P(Z > \dfrac{\ln K - a}{b}) = P(Z < \dfrac{a - \ln K}{b})\\
    & = \Phi \left(\dfrac{a - \ln K}{b}\right) = \Phi \left(\dfrac{\ln S_t + (r-\dfrac{1}{2}\sigma^2) (T-t) - \ln K}{\sigma\sqrt{T-t}}\right) = N(d_2)
\end{align*}
\begin{align*}
    E_2 = Ke^{-r(T-t)}E_Q[\ind{x>\ln K}|\mathcal{F}_t] = Ke^{-r(T-t)}N(d_2)
\end{align*}
\subsection*{Hedge parameters for Delta}
Delta is the hedge ratio of the option. $\delta = \dfrac{\partial C}{\partial S}$
\begin{align*}
    C_T = S N(d_1) - Ke^{-rT}N(d_2) 
\end{align*}
\begin{align*}
    \dfrac{\partial C}{\partial S} & = N(d_1) + S\dfrac{\partial N(d_1)}{\partial d_1}\dfrac{\partial d_1}{\partial S} - Ke^{-rT}\dfrac{\partial N(d_2)}{\partial d_2}\dfrac{\partial d_2}{\partial S}\\
    & = N(d_1) + S\dfrac{1}{\sqrt{2\pi}}\exp(-\dfrac{d_1^2}{2})\dfrac{\partial d_1}{\partial S} - Ke^{-rT}\dfrac{1}{\sqrt{2\pi}}\exp(-\dfrac{d_2^2}{2})\dfrac{\partial d_2}{\partial S}
\end{align*}
Because $d_2 = d_1 - \sigma\sqrt{T}$, and $\dfrac{\partial d_1}{\partial S} = \dfrac{\partial d_2}{\partial S}$
\begin{align*}
    \dfrac{\partial C}{\partial S} & = N(d_1) + S\dfrac{1}{\sqrt{2\pi}}\exp(-\dfrac{d_1^2}{2})\dfrac{\partial d_1}{\partial S} - Ke^{-rT}\dfrac{1}{\sqrt{2\pi}}\exp(-\dfrac{(d_1^2 - 2d_1 \sigma\sqrt{T} + \sigma^2 T)}{2})\dfrac{\partial d_1}{\partial S}\\
\end{align*}
As $d_1\sigma\sqrt{T} = \ln(\dfrac{S}{K})+(r+\dfrac{1}{2}\sigma^2)T$
\begin{align*}
    S\exp(-\dfrac{1}{2}d_1^2) &= Ke^{-rT}\exp(-\dfrac{1}{2}d_1^2 + \ln(\dfrac{S}{K})+(r+\dfrac{1}{2}\sigma^2)T - \dfrac{1}{2}\sigma^2T)\\
    & = Ke^{-rT}\exp(-\dfrac{1}{2}d_1^2 + \ln(\dfrac{S}{K}) + rT) = S\exp(-\dfrac{1}{2}d_1^2)
\end{align*}
So, at last
\begin{align*}
    \dfrac{\partial C}{\partial S} = N(d_1)
\end{align*}
\subsection*{Derive Merton's pricing PDE}
Same as 3-1, we first construct the risk-free investment portfolio $\pi_t = -f_t + \dfrac{\partial f_t}{\partial S_t}S_t$\\
In minimum time $dt$, change of the portfolio is
\begin{align*}
    d\pi_t & = -df_t + \dfrac{\partial f_t}{\partial S_t}(dS_t + qS_t dt)\\
    & = -\left(\pfpt + \pfps \mu S_t + \dfrac{1}{2}\pffpss \sigma^2 S_t^2\right)dt - \pfps \sigma S_tdW_t + \pfps (\mu S_t dt + \sigma S_t dW_t + qS_t dt)\\
    & = -\left(\pfpt + \dfrac{1}{2}\pffpss \sigma^2 S_t^2 - \pfps qS_t\right)dt = r\pi_t dt = r\left(-f_t + \dfrac{\partial f_t}{\partial S_t}S_t\right)dt
\end{align*}
So
\begin{align*}
    rf_t = \pfpt + \dfrac{1}{2}\pffpss \sigma^2 S_t^2 + \pfps (r-q)S_t
\end{align*}
\subsection*{Merton's pricing formula}
\begin{align*}
    C_t = Se^{-q(T-t)}N(d_1) - Ke^{-r(T-t)}N(d_2)
\end{align*}
Where $q$ is the dividend yield.\\
\section{Heat Equation}
\subsection*{Derive Black-Scholes PDE}
Solution of Heat Equation
\begin{align*}
    U(x,t) = \int_{-\infty}^{\infty} \dfrac{1}{\sqrt{4\pi c^2 t}}e^{-\dfrac{(x-y)^2}{4c^2t}}g(y)dy
\end{align*}
Where $U_t(x,t) = c^2 U_xx(x,t)$ subject to $U(x,0) = g(x)$
\begin{align*}
    V_{\tau}(x,\tau) = \int_{-\infty}^{\infty} \dfrac{1}{\sqrt{4\pi c^2 \tau}}e^{-\dfrac{(y-x)^2}{4c^2\tau}}K(e^y-1)^{+}dy
\end{align*}
Because $c^2 = \dfrac{1}{2}\sigma^2$
\begin{align*}
    V_{\tau}(x,\tau) & = K\int_{-\infty}^{\infty} \dfrac{1}{\sqrt{2\pi \sigma^2 \tau}}e^{-\dfrac{(y-x)^2}{2\sigma^2\tau}}(e^y-1)^{+}dy\\
    & = K\int_{0}^{\infty} \dfrac{1}{\sqrt{2\pi \sigma^2 \tau}}e^{-\dfrac{(y-x)^2}{2\sigma^2\tau}}(e^y-1)dy,\ y>0\leftrightarrow e^y>1\\
    & = K\left[\int_{0}^{\infty} \dfrac{1}{\sqrt{2\pi \sigma^2 \tau}}e^{-\dfrac{(y-x)^2}{2\sigma^2\tau}}e^y dy - \int_{0}^{\infty} \dfrac{1}{\sqrt{2\pi \sigma^2 \tau}}e^{-\dfrac{(y-x)^2}{2\sigma^2\tau}} dy\right]
\end{align*} 
Let $z = \dfrac{y-x}{\sigma\sqrt{\tau}}$, $y = \sigma\sqrt{\tau}\cdot z + x$, $dz = \dfrac{dy}{\sigma\sqrt{\tau}}$, $z^* = -\dfrac{x}{\sigma\sqrt{\tau}}=-d_2$
\begin{align*}
    V_{\tau}(x,\tau) & = K\left[\int_{-d2}^{\infty} \dfrac{1}{\sqrt{2\pi}}e^{-\frac{z^2}{2}}e^{\sigma\sqrt{\tau}\cdot z+x} dz - \int_{-d2}^{\infty} \dfrac{1}{\sqrt{2\pi}}e^{-\frac{z^2}{2}} dz\right]\\
    & = K\left[\int_{-d2}^{\infty} \dfrac{1}{\sqrt{2\pi}}e^{-\frac{(z-\sigma\sqrt{\tau})^2}{2}}e^{\frac{\sigma^2}{2}\tau+x} dz - \int_{-d2}^{\infty} \dfrac{1}{\sqrt{2\pi}}e^{-\frac{z^2}{2}} dz\right]\\
    & = Ke^{\frac{\sigma^2}{2}\tau+x}\int_{-d2}^{\infty} \dfrac{1}{\sqrt{2\pi}}e^{-\frac{(z-\sigma\sqrt{\tau})^2}{2}} dz - K\int_{-\infty}^{d2} \dfrac{1}{\sqrt{2\pi}}e^{-\frac{z^2}{2}} dz
\end{align*}
Let $z_1 = z - \sigma\sqrt{\tau}$, the lower bound of $z_1 = -d_1 = -d_2 - \sigma\sqrt{\tau}$
\begin{align*}
    V_{\tau}(x,\tau) & = Ke^{\frac{\sigma^2}{2}\tau+x}\int_{-d2-\sigma\sqrt{\tau}}^{\infty} \dfrac{1}{\sqrt{2\pi}}e^{-\frac{z_1^2}{2}} dz_1 - KN(d_2)\\
    & = Ke^{\frac{\sigma^2}{2}\tau+x}\int_{-\infty}^{d_1} \dfrac{1}{\sqrt{2\pi}}e^{-\frac{z_1^2}{2}} dz_1 - KN(d_2)\\
    & = Ke^{\frac{\sigma^2}{2}\tau+x}N(d_1) - KN(d_2),\ x = \ln\dfrac{S}{K} + (r-\dfrac{1}{2}\sigma^2)\tau\\
    & = Se^{r\tau}N(d_1) - KN(d_2)
\end{align*}
Last
\begin{align*}
    C(S_t, t) = e^{-r\tau}V(x,\tau) = SN(d_1) - Ke^{-r\tau}N(d_2)
\end{align*}
\section{Martingale Pricing Formula}
\subsection*{Describe the Fundamental Theorem of Arbitrage-Free}
Under the Arbitrage free condition, T exists an only risk neutral measurement $Q$ which let the relative price of asset follows Martingale.\\\\
That is,\\
\[\dfrac{H_t}{\beta_t} = E_Q\Bigg[\dfrac{H_T}{\beta_T}\Bigg|\mathcal{F}_t\Bigg]\]
\subsection*{Describe the Girsanov's Theorem}
If $E(e^{\int_0^t \beta_t dt})<\infty$. Let $P$ to $Q's$ Radon-Nikodym derivative $\xi_t = \dfrac{dP}{dQ}|\mathcal{F}_t = \exp\{\int_t^T X_s dW_s - \dfrac{1}{2}\int_t^T X_s^2 ds\}$ 
\subsection*{Derive Black-Scholes pricing formula by Martingale Pricing Method}
First, transfer dynamic stock price process of $P$ measure to $Q$ measure.\\
\begin{align*}
    \dfrac{dS_t}{S_t} = \mu dt + \sigma dW_t^P \to \dfrac{dS_t}{S_t} = r dt + \sigma dW_t^Q
\end{align*}
Second, We let the relative price of MMA follows Martingale under $Q$ measure.\\
\begin{align*}
    \dfrac{C_t}{\beta_t} = E\left[\dfrac{C_T}{\beta_T}|\mathcal{F}_t\right]
\end{align*}
We assume interest rate is constant
\begin{align*}
    C_t & = \dfrac{\beta_t}{\beta_T}E[C_T|\mathcal{F}_t]\\
    & = e^{-r(T-t)}E[C_T|\mathcal{F}_t]\\
    & = e^{-r(T-t)}E^Q(S_T\ind_{S_T>K}|\mathcal{F}_t) - Ke^{-r(T-t)}E^Q(\ind_{S_T>K}|\mathcal{F}_t)
\end{align*}
Expand the close form of Stock Price
\begin{align*}
    E^Q(S_T\ind_{S_T>K}|\mathcal{F}_t) & = E^Q(S_te^{(r - \frac{1}{2}\sigma^2)(T-t) + \sigma(W_T^Q - W_t^Q)}\ind_{S_T>K}|\mathcal{F}_t)\\
    & = S_te^{r(T-t)}E^Q(e^{(\frac{1}{2}\sigma^2)(T-t) + \sigma(W_T^Q - W_t^Q)}\ind_{S_T>K}|\mathcal{F}_t)\\
\end{align*}
Under Girsanov theorem
\begin{align*}
    \dfrac{dQ}{dP}|\mathcal{F}_t = \exp\left\{\int_t^T \sigma dW_s - \dfrac{1}{2}\int_t^T \sigma^2 ds\right\}
\end{align*}
So, we can derive the close form of stock price as
\begin{align*}
    E^Q(e^{(\frac{1}{2}\sigma^2)(T-t) + \sigma(W_T^Q - W_t^Q)}\ind_{S_T>K}|\mathcal{F}_t) & = E^Q(\dfrac{dP}{dQ}\cdot\ind_{S_T>K}|\mathcal{F}_t)\\
    & = E^P(\ind_{S_T>K}|\mathcal{F}_t)\\
    & = P^P(S_T>K|\mathcal{F}_t)\\
\end{align*}
\end{document}