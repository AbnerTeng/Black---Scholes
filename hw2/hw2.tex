\documentclass[12pt]{article}
 
\usepackage[margin=1in]{geometry} 
\usepackage{amsmath,amsthm,amssymb,scrextend}
\usepackage{fancyhdr}
\setlength{\headheight}{14.5pt}
\addtolength{\topmargin}{-2.5pt}
\pagestyle{fancy}

\newcommand{\cont}{\subseteq}
\usepackage{tikz}
\usepackage{pgfplots}
\usepackage{amsmath}
\usepackage[mathscr]{euscript}
\let\euscr\mathscr\let\mathscr\relax% just so we can load this and rsfs
\usepackage[scr]{rsfso}
\usepackage{amsthm}
\usepackage{amssymb}
\usepackage{multicol}
\usepackage[colorlinks=true, pdfstartview=FitV, linkcolor=blue,
citecolor=blue, urlcolor=blue]{hyperref}

\DeclareMathOperator{\arcsec}{arcsec}
\DeclareMathOperator{\arccot}{arccot}
\DeclareMathOperator{\arccsc}{arccsc}
\newcommand{\ddx}{\frac{d}{dx}}
\newcommand{\dfdx}{\frac{df}{dx}}
\newcommand{\ddxp}[1]{\frac{d}{dx}\left(#1 \right)}
\newcommand{\dydx}{\frac{dy}{dx}}
\let\ds\displaystyle
\newcommand{\intx}[1]{\int#1 \, dx}
\newcommand{\intt}[1]{\int#1 \, dt}
\newcommand{\defint}[3]{\int_{#1}^{#2} #3 \, dx}
\newcommand{\imp}{\Rightarrow}
\newcommand{\un}{\cup}
\newcommand{\inter}{\cap}
\newcommand{\ps}{\mathscr{P}}
\newcommand{\set}[1]{\left\{ #1 \right\}}
\newtheorem*{sol}{Solution}
\newtheorem*{claim}{Claim}
\newtheorem{problem}{Problem}
\pgfplotsset{compat=1.17}
\begin{document}
 
% Don't change the above session

\lhead{Financial Engineering hw2}
\chead{111352027}
\rhead{\today}
 
% \maketitle
\section{Proof properties of Wiener process}
Assume that $t<s$\\
\[
    X_s\sim N(0, s)\quad\ X_t\sim N(0, t)  
\]
\begin{enumerate}
    \item By independent increments, we know that $E(X_s|X_t=B) = E(X_s-X_t+X_t|X_t=B) = E(X_s-X_t|X_t=B)+X_t=0+B=B$
    \item $Var(X_s|X_t=B) = E\{[X_s-E(X_s|X_t=B)]^2|X_t=B\} = E[(X_s-B)^2|X_t=B]\\ = E[X_s^2-2X_s B+B^2|X_t=B]=E[X_s^2|X_t=B]-2BE(X_s|X_t=B)+B^2\\=E[X_s^2|X_t=B]-B^2=Var(X_s)-Var(X_t)=s-t$
    \item $X_t|X_s=B$ is a bivariate normal distribution which $\sim N(B\times\frac{t}{s}, (s-t)\times\frac{t}{s})$. So $E(X_t|X_s=B) = B\times\frac{t}{s}$
    \item According to 3. $Var(X_t|X_s=B) = (s-t)\times\frac{t}{s}$
    \item $Cov(X_s, X_t) = E(X_s X_t) - E(X_s)E(X_t) = E(X_s X_t)-0 = E(X_s X_t)\\ = E(X_s-X_{s-1}+X_{s-1}-X_{s-2}+\cdots+X_{t+1}-X_{t}+X_t)X_t  \\ = E(X_S-X_{s-1})X_t + \cdots +E(X_t)X_t = E(X_t^2) = t^2$
\end{enumerate}


\section{Proof the Quadratic Variation of Brownian Motion}
According to standard Brownian Motion
\[
    dX_t = adt + bdW_t  
\]
By definition, $(dZ)^2 = \varepsilon^2 dt,\ \forall \varepsilon\sim N(0, 1)$. So
\[
    Var(\varepsilon) = 1 = E[\varepsilon^2] \Rightarrow E[(dZ)^2] = E[\varepsilon^2 dt] = 1dt = dt  
\]
In addition, $Var((dZ^2)) = Var(\varepsilon^2 dt) = (dt^2)Var(\varepsilon) \to 0$ (because $(dt^2\to 0)$). So we say
\[
    (dZ)^2 \stackrel{a.s.}{=} dt
\]  
Based on the result of $(dZ)^2 = dt$, we can infer that the quadratic variation of Brownian Motion over $[0, T]$, i.e., $[W,W](t) = \lim_{n\to\infty}\sum_{i=1}^n|B(t_i^n)-B(t_{i-1}^n)|^2=T$.\\\\
<Proof>\\
At first we know $(dZ)^2\sim N(dt, 0)$ when $n\to\infty$ or $dt\to 0$, $(B(t_i^n)-B(t_{i-1}^n))\stackrel{a.s.}{=}(t_i^n-t_{i-1}^n)$.\\
This is because $E[(B(t_i^n)-B(t_{i-1}^n))^2] = E[\varepsilon^2(t_i^n-t_{i-1}^n)] = t_i^n - t_{i-1}^n$, and\\
$Var((B(t_i^n)-B(t_{i-1}^n))^2) = Var(\varepsilon^2(t_i^n-t_{i-1}^n)) = (t_i^n - t_{i-1}^n)^2Var(\varepsilon^2)\to 0$.\\\\
Thus, we can conclude that when $n\to \infty$, or $t_i^n-t_{i-1}^n \to 0$, $\lim_{n\to\infty}\sum_{i=1}^n|B(t_i^n)-B(t_{i-1}^n)|^2=T$






\newpage
\section{Application of It\^o's Lemma (I)}
Assume the stock price change process is
\[
    dS_t = \mu S_t dt + \sigma S_t dz_t  
\]
\[
    f_t = \ln S_t
\]  
According to Taylor series
\begin{align*}
    df_t = (\frac{\partial f}{\partial t}dt+\frac{\partial f}{\partial S}\mu S_t(S_t, t)+\frac{1}{2}\frac{\partial^2f}{\partial S^2}\sigma^2 S_t^2(S_t, t))dt + \frac{\partial f}{\partial S}\sigma S_t(S_t, t)dz_t
\end{align*}
Because $f_t = \ln S_t$
\begin{align*}
    d\ln S_t &= (\frac{1}{S_t}\mu S_t + \frac{-1}{2S_t^2}\sigma^2S_t^2)dt + \frac{1}{S_t}\sigma S_t dz_t\\
    &= (\mu - \frac{1}{2}\sigma^2)dt + \sigma dz_t
\end{align*}
Then we inergrate it to get the close form of stock price.
\begin{align*}
   \int_0^T d\ln S_t &= \int_0^T (\mu - \frac{1}{2}\sigma^2)dt + \int_0^T \sigma dz_t\\
   \ln(S_T)-\ln(S_0) &= \ln(\dfrac{S_T}{S_0}) = (\mu-\frac{1}{2}\sigma^2)T + \sigma z_T\\
   \dfrac{S_T}{S_0} &= \exp{[(\mu-\frac{1}{2}\sigma^2)T + \sigma z_T]}\\
   S_T &= S_0\exp{[(\mu-\frac{1}{2}\sigma^2)T + \sigma z_T]} 
\end{align*}
If $S_T$ is Logarithm Normal Distributed, then $\ln(S_T)$ is Normal Distributed.\\ 
\begin{align*}
    \ln(S_T) &= \ln(S_0) + (\mu-\frac{1}{2}\sigma^2)T + \sigma z_T\sim N(\ln(S_0) + (\mu-\frac{1}{2}\sigma^2)T, \sigma^2 T)\\
    E(S_T) &= \exp[\ln(S_0) + (\mu-\frac{1}{2}\sigma^2)T + \frac{1}{2}\sigma^2 T] = S_0e^{\mu T}\\
    E(S_T^2) &= \exp[2\ln(S_0) + 2(\mu-\frac{1}{2}\sigma^2)T + 2\frac{1}{2}\sigma^2 T] = S_0^2e^{2\mu T + \sigma^2 T}\\
    Var(S_T) &= E(S_T^2) - [E(S_T)]^2 = S_0^2(e^{2\mu T + \sigma^2 T} - e^{2\mu T})
\end{align*}
\newpage

\section{Application of It\^o's Lemma (II)}
Assume the stock price change process is
\[
    dS_t = \mu S_t dt + \sigma S_t dz_t  
\]
\[
    f_t = S_t - Ke^{-r(T-t)}
\]  
According to Taylor series
\begin{align*}
    df_t &= (-rK^{-r(T-t)} + \mu S_t + \frac{(\sigma S_t)^2}{2}\times 0)dt + \sigma S_t dz_t\\
    &= (\mu S_t - rK^{-r(T-t)})dt + \sigma S_t dz_t
\end{align*}

\section{Application of It\^o's Lemma (III)}
Assume the interest rate change process is
\[
    dr_t = \kappa(\theta - r_t)dt + \sigma dz_t
\]
\[
    f_t = r_t e^{\kappa t}
\]  
According to Taylor series
\begin{align*}
    df_t &= (\frac{\partial f}{\partial t} + \frac{\partial f}{\partial r}\kappa(\theta-r_t) + \frac{1}{2}\frac{\partial^2f}{\partial r^2}\sigma^2(r_t, t))dt + \frac{\partial f}{\partial r}\sigma(r_t, t)dz_t\\
    &= (\kappa r_t e^{\kappa t} + e^{\kappa t}\kappa (\sigma-r_t))dt + e^{\kappa t}\sigma dz_t
\end{align*}
Because $f_t = r_t e^{\kappa t}$
\begin{align*}
    dr_t e^{\kappa t} &= (\kappa r_t e^{\kappa t} + e^{\kappa t}\kappa (\sigma-r_t))dt + e^{\kappa t}\sigma dz_t\\
    dr_t &= (\kappa r_t + \kappa (\sigma-r_t))dt + \sigma dz_t\\
    &= \kappa\sigma dt + \sigma dz_t
\end{align*}
Then we inergrate it to get the close form of interest rate.
\begin{align*}
    \int_0^T dr_t &= \int_0^T \kappa\sigma dt + \int_0^T \sigma dz_t\\
    r_T -r_0 &= \int_0^T \kappa\sigma dt + \int_0^T \sigma dz_t\\
    r_T &= r_0 + \kappa\sigma T + \sigma z_T
\end{align*}
If $r_T$ is Normal Distributed,
\begin{align*}
    r_T &= r_0 + \kappa\sigma T + \sigma z_T\sim N(r_0 + \kappa\sigma T, \sigma^2 T)\\
    E(r_T) &= r_0 + \kappa\sigma T\\
    E(r_T^2) &= r_0^2 + 2r_0\kappa\sigma T + \kappa^2\sigma^2 T\\
    Var(r_T) &= E(r_T^2) - [E(r_T)]^2 = r_0^2 + 2r_0\kappa\sigma T + \kappa^2\sigma^2 T - (r_0 + \kappa\sigma T)^2 = 0
\end{align*}
\newpage
\section{Application of It\^o's Lemma (IV)}
Assume the stock price change process and foriegn exchange rate change process are
\begin{align*}
    dS_t = \mu S_t dt + \sigma S_t dz_{St}\\
    dX_t = \mu X_t dt + \sigma X_t dz_{Xt}\\
    Cov(dz_{St}, dz_{Xt}) = \rho dt
\end{align*}
\subsection{$f_t = S_t X_t$}
if $f_t = f(t, X_t, S_t) = S_t X_t$, According to Taylor series
\begin{align*}
    df_t &= (\frac{\partial f_t}{\partial t} + \frac{\partial f_t}{\partial S_t}\mu_S S_t + \frac{\partial f_t}{\partial X_t}\mu_X X_t + \frac{1}{2}\frac{\partial^2 f_t}{\partial S_t^2}\sigma^2_S S_t^2 + \frac{1}{2}\frac{\partial^2 f_t}{\partial X_t^2}\sigma^2_X X_t^2 + \frac{\partial^2 f_t}{\partial S_t \partial X_t}\sigma_S S_t \sigma_X X_t \rho)dt \\
    &+ \frac{\partial f_t}{\partial S_t}\sigma_S S_t dz_{St} + \frac{\partial f_t}{\partial X_t}\sigma_X X_t dz_{Xt}\\
    &= (X_t\mu_S S_t + S_t \mu_X X_t + \sigma_S\sigma_X S_t X_t\rho)dt + X_t \sigma_S S_t dz_{St} + S_t \sigma_X X_t dz_{Xt}\\
\end{align*}
Because $f_t = S_t X_t$
\begin{align*}
    dS_t X_t = (X_t\mu_S S_t + S_t \mu_X X_t + \sigma_S\sigma_X S_t X_t\rho)dt + X_t \sigma_S S_t dz_{St} + S_t \sigma_X X_t dz_{Xt} 
\end{align*}
Then we intergrate it to get the close form of stock price with NTD dollars.
\begin{align*}
    \int_o^T dS_t X_t &= \int_0^T (X_t\mu_S S_t + S_t \mu_X X_t + \sigma_S\sigma_X S_t X_t\rho)dt + \int_o^T X_t \sigma_S S_t dz_{St} + \int_o^T S_t \sigma_X X_t dz_{Xt}\\
    S_T X_T-S_0 X_0 &= \int_0^T(S_t X_t(\mu_S + \mu_X+\sigma_S\sigma_X\rho))dt + \int_0^T S_t X_t(\sigma_S dz_{St} + \sigma_X dz_{Xt})\\
    &= [(S_T X_T - S_0 X_0)(\mu_S + \mu_X+\sigma_S\sigma_X\rho)]T + S_T X_T(\sigma_S dz_{ST} + \sigma_X dz_{XT})\\\\
    S_T X_T &= S_0 X_0 + [(S_T X_T - S_0 X_0)(\mu_S + \mu_X+\sigma_S\sigma_X\rho)]T + S_T X_T(\sigma_S dz_{ST} + \sigma_X dz_{XT})
\end{align*}
\subsection{$f_t = \dfrac{S_t}{X_t}$}
if $f_t = f(t, X_t, S_t) = \dfrac{S_t}{X_t}$, According to Taylor series
\begin{align*}
    df_t &= (\frac{\partial f_t}{\partial t} + \frac{\partial f_t}{\partial S_t}\mu_S S_t + \frac{\partial f_t}{\partial X_t}\mu_X X_t + \frac{1}{2}\frac{\partial^2 f_t}{\partial S_t^2}\sigma^2_S S_t^2 + \frac{1}{2}\frac{\partial^2 f_t}{\partial X_t^2}\sigma^2_X X_t^2 + \frac{\partial^2 f_t}{\partial S_t \partial X_t}\sigma_S S_t \sigma_X X_t \rho)dt \\
    &+ \frac{\partial f_t}{\partial S_t}\sigma_S S_t dz_{St} + \frac{\partial f_t}{\partial X_t}\sigma_X X_t dz_{Xt}\\
    &= (\dfrac{1}{X_t}\mu_S S_t - \dfrac{S_t}{X_t} \mu_X + 2\dfrac{S_t}{X_t}\sigma^2_X -\dfrac{1}{X_t}\sigma_S S_t \sigma_X\rho)dt + \dfrac{1}{X_t} \sigma_S S_t dz_{St} -\dfrac{S_t}{X_t} \sigma_X dz_{Xt}\\
\end{align*}
Because $f_t = \dfrac{S_t}{X_t}$
\begin{align*}
    d\dfrac{S_t}{X_t} = (\dfrac{1}{X_t}\mu_S S_t - \dfrac{S_t}{X_t} \mu_X + 2\dfrac{S_t}{X_t}\sigma^2_X -\dfrac{1}{X_t}\sigma_S S_t \sigma_X\rho)dt + \dfrac{1}{X_t} \sigma_S S_t dz_{St} -\dfrac{S_t}{X_t} \sigma_X dz_{Xt}
\end{align*}
Then we intregrate it to get the close form of stock price with NTD dollars.
\begin{align*}
    \int_o^T d\dfrac{S_t}{X_t} &= \int_0^T (\dfrac{1}{X_t}\mu_S S_t - \dfrac{S_t}{X_t} \mu_X + 2\dfrac{S_t}{X_t}\sigma^2_X -\dfrac{1}{X_t}\sigma_S S_t \sigma_X\rho)dt + \int_o^T \dfrac{1}{X_t} \sigma_S S_t dz_{St} -\int_o^T \dfrac{S_t}{X_t} \sigma_X dz_{Xt}\\
    \dfrac{S_T}{X_T}-\dfrac{S_0}{X_0} &= \int_0^T(\dfrac{1}{X_t}\mu_S S_t - \dfrac{S_t}{X_t} \mu_X + 2\dfrac{S_t}{X_t}\sigma^2_X -\dfrac{1}{X_t}\sigma_S S_t \sigma_X\rho)dt + \int_0^T \dfrac{1}{X_t} \sigma_S S_t dz_{St} -\int_0^T \dfrac{S_t}{X_t} \sigma_X dz_{Xt}\\
    &= [(\dfrac{S_T}{X_T}-\dfrac{S_0}{X_0})(\mu_S - \mu_X + 2\sigma^2_X -\sigma_S\sigma_X\rho)]T + \dfrac{S_T}{X_T}(\sigma_S dz_{ST} - \sigma_X dz_{XT})\\\\
    \dfrac{S_T}{X_T} &= \dfrac{S_0}{X_0} + [(\dfrac{S_T}{X_T}-\dfrac{S_0}{X_0})(\mu_S - \mu_X + 2\sigma^2_X -\sigma_S\sigma_X\rho)]T + \dfrac{S_T}{X_T}(\sigma_S dz_{ST} - \sigma_X dz_{XT})
\end{align*}
\end{document}