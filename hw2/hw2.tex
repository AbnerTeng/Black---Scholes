\documentclass[12pt]{article}
 
\usepackage[margin=1in]{geometry} 
\usepackage{amsmath,amsthm,amssymb,scrextend}
\usepackage{fancyhdr}
\setlength{\headheight}{14.5pt}
\addtolength{\topmargin}{-2.5pt}
\pagestyle{fancy}

\newcommand{\cont}{\subseteq}
\usepackage{tikz}
\usepackage{pgfplots}
\usepackage{amsmath}
\usepackage[mathscr]{euscript}
\let\euscr\mathscr\let\mathscr\relax% just so we can load this and rsfs
\usepackage[scr]{rsfso}
\usepackage{amsthm}
\usepackage{amssymb}
\usepackage{multicol}
\usepackage[colorlinks=true, pdfstartview=FitV, linkcolor=blue,
citecolor=blue, urlcolor=blue]{hyperref}

\DeclareMathOperator{\arcsec}{arcsec}
\DeclareMathOperator{\arccot}{arccot}
\DeclareMathOperator{\arccsc}{arccsc}
\newcommand{\ddx}{\frac{d}{dx}}
\newcommand{\dfdx}{\frac{df}{dx}}
\newcommand{\ddxp}[1]{\frac{d}{dx}\left(#1 \right)}
\newcommand{\dydx}{\frac{dy}{dx}}
\let\ds\displaystyle
\newcommand{\intx}[1]{\int#1 \, dx}
\newcommand{\intt}[1]{\int#1 \, dt}
\newcommand{\defint}[3]{\int_{#1}^{#2} #3 \, dx}
\newcommand{\imp}{\Rightarrow}
\newcommand{\un}{\cup}
\newcommand{\inter}{\cap}
\newcommand{\ps}{\mathscr{P}}
\newcommand{\set}[1]{\left\{ #1 \right\}}
\newtheorem*{sol}{Solution}
\newtheorem*{claim}{Claim}
\newtheorem{problem}{Problem}
\pgfplotsset{compat=1.17}
\begin{document}
 
% Don't change the above session

\lhead{Financial Engineering hw2}
\chead{111352027}
\rhead{\today}
 
% \maketitle
\section{Proof properties of Wiener process}
Assume that $t<s$\\
\[
    X_s\sim N(0, s)\quad\ X_t\sim N(0, t)  
\]
\begin{enumerate}
    \item By independent increments, we know that $E(X_s|X_t=B) = E(X_s-X_t+X_t|X_t=B) = E(X_s-X_t|X_t=B)+X_t=0+B=B$
    \item $Var(X_s|X_t=B) = E\{[X_s-E(X_s|X_t=B)]^2|X_t=B\} = E[(X_s-B)^2|X_t=B]\\ = E[X_s^2-2X_s B+B^2|X_t=B]=E[X_s^2|X_t=B]-2BE(X_s|X_t=B)+B^2\\=E[X_s^2|X_t=B]-B^2=Var(X_s)-Var(X_t)=s-t$
    \item $X_t|X_s=B$ is a bivariate normal distribution which $\sim N(B\times\frac{t}{s}, (s-t)\times\frac{t}{s})$. So $E(X_t|X_s=B) = B\times\frac{t}{s}$
    \item According to 3. $Var(X_t|X_s=B) = (s-t)\times\frac{t}{s}$
    \item $Cov(X_s, X_t) = E(X_s X_t) - E(X_s)E(X_t) = E(X_s X_t)-0 = E(X_s X_t)\\ = E(X_s-X_{s-1}+X_{s-1}-X_{s-2}+\cdots+X_{t+1}-X_{t}+X_t)X_t  \\ = E(X_S-X_{s-1})X_t + \cdots +E(X_t)X_t = E(X_t^2) = t^2$
\end{enumerate}


\section{Proof the Quadratic Variation of Brownian Motion}
Proof.


\newpage
\section{Application of It\^o's Lemma (I)}
Assume the stock price change process is
\[
    dS_t = \mu S_t dt + \sigma S_t dz_t  
\]
\[
    f_t = \ln S_t
\]  
According to Taylor series
\begin{align*}
    df = (\frac{\partial f}{\partial t}dt+\frac{\partial f}{\partial S}\mu S_t(S_t, t)+\frac{1}{2}\frac{\partial^2f}{\partial S^2}\sigma^2 S_t^2(S_t, t))dt + \frac{\partial f}{\partial S}\sigma S_t(S_t, t)dz_t
\end{align*}
Because $f_t = \ln S_t$
\begin{align*}
    d\ln S_t &= (\frac{1}{S_t}\mu S_t + \frac{-1}{2S_t^2}\sigma^2S_t^2)dt + \frac{1}{S_t}\sigma S_t dz_t\\
    &= (\mu - \frac{1}{2}\sigma^2)dt + \sigma dz_t
\end{align*}
Then we inergrate it to get the close form of stock price.
\begin{align*}
   \int_0^T d\ln S_t = \int_0^T (\mu - \frac{1}{2}\sigma^2)dt + \int_0^T \sigma dz_t\\
   \ln(S_T-S_0) = (\mu-\frac{1}{2}\sigma^2)T + \sigma z_T\\
   (S_T-S_0) = \exp{[(\mu-\frac{1}{2}\sigma^2)T + \sigma z_T]}\\
   S_T = S_0 + \exp{[(\mu-\frac{1}{2}\sigma^2)T + \sigma z_T]} 
\end{align*}
Expectation value and variance
\newpage

\section{Application of It\^o's Lemma (II)}
Assume the stock price change process is
\[
    dS_t = \mu S_t dt + \sigma S_t dz_t  
\]
\[
    f_t = S_t - Ke^{-r(T-t)}
\]  
According to Taylor series
\begin{align*}
    df &= (-rK^{-r(T-t)} + \mu S_t + \frac{(\sigma S_t)^2}{2}\times 0)dt + \sigma S_t dz_t\\
    &= (\mu S_t - rK^{-r(T-t)})dt + \sigma S_t dz_t
\end{align*}

\section{Application of It\^o's Lemma (III)}
Assume the stock price change process is
\[
    dr_t = \kappa   
\]
\[
    f_t = \ln S_t
\]  
 
\end{document}